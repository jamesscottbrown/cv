% LaTeX Curriculum Vitae Template
%
% Copyright (C) 2004-2008 Jason Blevins <jrblevin@sdf.lonestar.org>
% http://jblevins.org/projects/cv-template
%
% You may use use this document as a template to create your own CV
% and you may redistribute the source code freely. No attribution is
% required in any resulting documents. I do ask that you please leave
% this notice and the above URL in the source code if you choose to
% redistribute this file.

\documentclass[letterpaper]{article}

\usepackage{color}
\usepackage{geometry}
\usepackage[T1]{fontenc}
\usepackage{mathpazo}
\usepackage{hyperref}

\def\name{James Scott-Brown}

% Make it tighter:
\setlength{\parskip}{5pt}
\setlength{\parsep}{0pt}
\setlength{\partopsep}{0pt}
\usepackage[compact]{titlesec}
\usepackage{paralist}
\addtolength{\itemsep}{-10mm}

% The following metadata will show up in the PDF properties
%\hypersetup{
%  colorlinks = true,
%  urlcolor = black,
%  pdfauthor = {\name},
%  pdfkeywords = {},
%  pdftitle = {\name: Curriculum Vitae},
%  pdfsubject = {Curriculum Vitae},
%  pdfpagemode = UseNone
%}

\geometry{a4paper, textwidth=6.5in, textheight=10in, marginparsep=7pt, marginparwidth=.6in}

% Customize page headers
\pagestyle{myheadings}
\markright{\name}
\thispagestyle{empty}

% Customize section headings
\usepackage{sectsty}
\sectionfont{\rmfamily\mdseries\Large}
\subsectionfont{\rmfamily\mdseries\scshape\normalsize}

% Don't indent paragraphs.
\setlength\parindent{0em}

% Define \compactitem to make lists without bullets
\renewenvironment{itemize}{
  \begin{compactitem}{}{
    \setlength{\leftmargin}{1.5em}
  }
}{
  \end{compactitem}
}

% Don't number sections
\setcounter{secnumdepth}{0}

% Define \years to put years in margin
% from http://nitens.org/taraborelli/cvtex
\usepackage{marginnote}
\newcommand{\years}[1]{\marginnote{\bf \small #1}}
\renewcommand*{\raggedleftmarginnote}{}
\setlength{\marginparsep}{0 pt}
\reversemarginpar

\newcommand{\nextItem}{\\[0.4\baselineskip]}
\newcommand{\nextItemTight}{\\[0.2\baselineskip]}


\begin{document}

% Place name at left
{\huge \name}

% Alternatively, print name centered and bold:
%\centerline{\huge \bf \name}

\vspace{0.25in}
\begin{minipage}[t]{0.5\textwidth}
\begin{tabular}{l}  	
Synthetic Biology Centre for Doctoral Training,\\
University of Oxford\\
http://jamesscottbrown.com\\
james.scott-brown@dtc.ox.ac.uk \\ \\
\end{tabular}
\end{minipage}
\begin{minipage}[t]{0.5\textwidth}
\begin{tabular}{ll}  	
Date of Birth: &October 19, 1991 \\
Citizenship: &British \\
ORCID: &\href{https://orcid.org/0000-0001-5642-8346}{0000-0001-5642-8346} \\ \\
\end{tabular}
\end{minipage}

\href{http://sysos.eng.ox.ac.uk/tebio/}{Research Homepage}

I am currently a D.Phil student on the 4-year EPSRC/BBSRC Synthetic
Biology Centre for Doctoral Training program, supervised by
\textbf{Prof.~Antonis Papachristodoulou} (Department of Engineering
Science, University of Oxford).

The broad goal of my research is to develop a design framework
(including associated algorithms and software) for designing biological
circuits at the population level. A major focus is on solving
visualisation and human-computer interaction problems that arise when
trying to do this.

\subsection{Research Interests}\label{research-interests}

\begin{itemize}
\tightlist
\item
  Applications of control theory to Synthetic Biology
\item
  Relationship between natural and engineered systems
\item
  `Design principles' of living systems
\item
  Visualization and Human-Computer Interaction
\end{itemize}

\subsection{Education}\label{education}

\begin{itemize}
\item
  BA and M.Eng (Cantab) 2014
\item
  2013-14: Part IIB of the Engineering Tripos (Merit). Completed project
  on Computational Auditory Scene Analysis, supervised by Dr.~Rich
  Turner. Studied the modules:

  Robust and non-linear systems and control, Optimal and predictive
  control, Practical optimisation, Statistical Pattern Processing,
  Machine Learning, Computational neuroscience, Molecular modelling,
  Accounting and finance.
\item
  2012-12: Part IIA of the Engineering Tripos (Class I). Studied the
  modules:

  Signals and systems, Systems and Control, Computer and Network
  Systems, Software Engineering and Design, Medical Imaging and 3D
  Computer Graphics, Mathematical Physiology, Introduction to
  Neuroscience, Introduction to Molecular Bioengineering, Biomaterials,
  Operations Management for Engineers
\item
  2011-12: Part I of the Natural Sciences Tripos (Class II). Studied
  Biology of Cells, Physiology of Organisms, Chemistry, and Mathematics
  B at Part IA. Studied Biochemistry \& Molecular Biology, Cell \&
  Developmental Biology, and Mathematics at Part IB.
\item
  A-levels in Physics, Chemistry, Biology, Mathematics and Further
  Mathematics.
\end{itemize}

\subsection{Publications (peer-reviewed journal
papers)}\label{publications-peer-reviewed-journal-papers}

\begin{itemize}
\item
  \textbf{James Scott-Brown}, Antonis Papachristodoulou. Visual
  representation of experimental protocols. \emph{In preparation}.
\item
  Melis Kayikci, AJ Venkatakrishnan, \textbf{James Scott-Brown}, Charles
  Ravarani, Mr.~Tilman Flock, M. Madan Babu. Visualization. Analysis of
  non-covalent contacts in biomolecules. \emph{Under review}.
\item
  \textbf{James Scott-Brown}, Antonis Papachristodoulou. sbml-diff: A
  tool for visually comparing SBML models in synthetic biology. ACS
  Synthetic Biology. Dec 2016. DOI:
  \href{http://dx.doi.org/10.1021/acssynbio.6b00273}{10.1021/acssynbio.6b00273}
\item
  \textbf{James Scott-Brown}, O A Cunningham, B C Goad. How hot can a
  fire piston get?, \emph{Physics Education} 45, 2010, p.32
  \href{http://dx.doi.org/10.1088/0031-9120/45/4/F04}{DOI:10.1088/0031-9120/45/4/F04}
\end{itemize}

\subsection{Talks}\label{talks}

\begin{itemize}
\item
  December 2017: \textbf{invited} to give an oral presentation at the
  IET Engineering of Biology in London, UK.
\item
  August 2017: orally presented an abstract about interfaces for
  expressing laboratory protocols at the 9th International Workshop on
  Bio-Design Automation (IWBDA) in Pittsburg, USA.
\item
  August 2016: orally presented an abstract about visually presenting
  and comparing SBML models at the 8th International Workshop on
  Bio-Design Automation (IWBDA) in Necastle, UK.
\end{itemize}

\subsection{Posters}\label{posters}

\begin{itemize}
\tightlist
\item
  \emph{Visualization of Temporal Logic Specifications}, Eurovis, June
  2017.
\item
  \emph{Visual Comparison of SBML Models}, Computational Modeling in
  Biology (COMBINE), September 2016
\end{itemize}

\subsection{Professional service}\label{professional-service}

\begin{itemize}
\tightlist
\item
  Maintain \href{http://visperception.com}{visperception.com}, a
  bibliography of experimental studies of the perception of
  visualizations
\end{itemize}

\subsection{Projects}\label{projects}

\begin{itemize}
\item
  2014: \textbf{DTC mini-project 1}, on `External Control of Gene
  Expression', supervised by \textbf{Prof.~Mario di Bernardo}
  (Department of Engineeirng Mathematics, University of Bristol). This
  involved designing and simulating controllers that would be
  implemented in software and control living cells in a microfluidic
  device.
\item
  2014: \textbf{DTC mini-project 2}, on `Applications of cell-cell
  communication to Synthetic Biology', supervised by
  \textbf{Prof.~Antonis Papachristodoulou} (Department of Engineering
  Science, University of Oxford). This involved modelling novel
  synthetic circuits that exploited quorum sensing systems.
\item
  2013: Software Developer Intern at \textbf{The MathWorks}. Prototyped
  a web GUI using JavaScript and the Dojo toolkit. Also implemented a
  new graph layout and improved GUI for a dependency graph viewer.
\item
  2012: Student placement with \textbf{Dr Madan Babu Mohan's} group at
  the MRC Laboratory of Molecular Biology. Constructed a website to
  visualise and compare networks of non-covalent interactions within
  proteins. I implemented the front-end in JavaScript with D3.js, and
  the backend in C++.
\item
  2011: Student placement with \textbf{Dr Bill Schafer's} group at the
  MRC Laboratory of Molecular Biology. Wrote a MATLAB program to play
  videos of experiments, and interact with metadata and annotations
  stored in both a MySQL database and binary files. I also had some
  exposure to \emph{C. elegans} experimental methods, including
  behavioural assays and calcium imaging.
\item
  2010: Student placement with \textbf{Prof.~Jeremy Frey's} group at the
  University of Southampton. This was mostly spent writing a Ruby on
  Rails web application to collect experimental data.
\item
  2008: Student placement with \textbf{Prof.~Jeremy Frey's} group. This
  included writing Perl scripts as part of a laboratory automation
  project.
\end{itemize}

\subsection{Hobbies}\label{hobbies}

\begin{itemize}
\tightlist
\item
  Sailing
\item
  Reading
\item
  Cooking
\item
  Playing with computers (programming, system administration, etc.)
\end{itemize}

\vspace{0.3 cm}
% References available on request.

\end{document}